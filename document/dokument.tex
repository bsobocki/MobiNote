% Opcje klasy 'iithesis' opisane sa w komentarzach w pliku klasy. Za ich pomoca
% ustawia sie przede wszystkim jezyk oraz rodzaj (lic/inz/mgr) pracy.
\documentclass[shortabstract]{iithesis}

\usepackage[utf8]{inputenc}
\usepackage{hyperref}
\usepackage{graphicx}
\usepackage{paralist}

%%%%% DANE DO STRONY TYTUŁOWEJ
% Niezaleznie od jezyka pracy wybranego w opcjach klasy, tytul i streszczenie
% pracy nalezy podac zarowno w jezyku polskim, jak i angielskim.
% Pamietaj o madrym (zgodnym z logicznym rozbiorem zdania oraz estetyka) recznym
% zlamaniu wierszy w temacie pracy, zwlaszcza tego w jezyku pracy. Uzyj do tego
% polecenia \fmlinebreak.
\polishtitle    {Aplikacja mobilna do zarządzania i tworzenia\fmlinebreak  interaktywnych notatek}
\englishtitle   {A mobile application to manage and create interactive notes}
\polishabstract {TODO: streszczenie po polsku\ldots}
\englishabstract{TODO: streszczenie po angielsku\ldots}
% w pracach wielu autorow nazwiska mozna oddzielic poleceniem \and
\author         {Bartosz Sobocki}
% w przypadku kilku promotorow, lub koniecznosci podania ich afiliacji, linie
% w ponizszym poleceniu mozna zlamac poleceniem \fmlinebreak
\advisor        {dr Marcin Młotkowski}
%\date          {}                     % Data zlozenia pracy
% Dane do oswiadczenia o autorskim wykonaniu
%\transcriptnum {}                     % Numer indeksu
%\advisorgen    {dr. Marcina Młotkowskiego} % Nazwisko promotora w dopelniaczu
%%%%%

%%%%% WLASNE DODATKOWE PAKIETY
%
%\usepackage{graphicx,listings,amsmath,amssymb,amsthm,amsfonts,tikz}
%
%%%%% WŁASNE DEFINICJE I POLECENIA
%
%\theoremstyle{definition} \newtheorem{definition}{Definition}[chapter]
%\theoremstyle{remark} \newtheorem{remark}[definition]{Observation}
%\theoremstyle{plain} \newtheorem{theorem}[definition]{Theorem}
%\theoremstyle{plain} \newtheorem{lemma}[definition]{Lemma}
%\renewcommand \qedsymbol {\ensuremath{\square}}
% ...
%%%%%

\begin{document}

%%%%% POCZĄTEK ZASADNICZEGO TEKSTU PRACY

\chapter{Wprowadzenie}

\section{Cel projektu}

Celem projektu jest wytworzenie aplikacji mobilnej do zarządzania notatkami w interaktywny sposób.
Aplikacja ma umożliwić użytkownikowi prowadzenie i tworzenie notatek pozwalających na edycję tekstu i inncyh elementów znajdujących się w notatce bez konieczności zmian widoków, przełączenia całej zawartości pomiędzy trybem edycji, a trybem użytkowym, czy potrzeby zapisu notatki, aby jej zawartość była dostępna w postaci końcowego efektu.
Graficzny interfejs użytkownika powinien być przystępny i pozwalać na edycję poszczególnych elementów, podczas gdy pozostała zawartość notatki jest wyświetlana w trybie widoku.
Do przechowywania zawartości notatek powinna zostać użyta baza danych, jak również odpowiedni format zapisu danych do przechowywania zawartości niebędącej tekstem.

\section{Motywacja}

Motywacją do stworzenia projektu MobiNote była potrzeba aplikacji, która służyłaby mi za brudnopis podczas wielu codziennych czynności. Jestem osobą, która uwielbia tworzyć notatki, prowadzić dzienniki, zapisywać przepisy, tworzyć wszelakie listy, dzielić problemy na mniejsze części i rozpisywać je, jak również wiele innych czynności związanych z prowadzeniem zeszytów. Dzięki temu mogę zawsze sięgać do moich zapisków po moje myśli, pomysły, orgzanizacje w chwilach, gdy są mi potrzebne.

Dostępne na rynku aplikacje, takie jak Evernote, czy ostatnio używane przeze mnie, proste w użyciu Samsung Notes w wielu aspektach sprawdziło się i przyniosło wiele korzyści. W obu przypadkach interfejs użytkownika jest przejrzysty, a funkcjonalnośc obszerna. Jednak po dłuższym okresie użytkowania tych, oraz innych aplikacji dostrzegam problemy lub braki, które chciałbym w jak największym stopniu wyeliminować.

Z upodobania do języka znaczników Markdown i tego, jakie możliwości w edycji stylów, czy tworzenia różnych nagłówków i elementów notatki oferuje poprzez samą składnię. W przeciwnieństwie do omawianych wyżej aplikacji nie ma potrzeby przeszukiwania opcji w poszukiwaniu przycisków, którymi użytkownik określa typ i wielkość czcionki, przełączania klawiatury na tę zawierającą narzędzia do edycji tekstu, czy wchodzenia w menu aby zaimportować zdjęcia, dodać listy itd. Dużym minusem jednak był brak aplikacji mobilnej, oraz dwa ekrany: edycji tekstu i wyrenderowanej notatki. Zależało mi na tym, aby notatka była renderowana i edytowana na bieżąco z ewentualnym ukazywaniem i chowaniem znaczników w tekście. Tekst powinien być stylizowany już w polu tekstowym uwzględniając dodawanie i usuwanie znaczników w czasie rzeczywistym.

Chciałem stworzyć aplikację, która będzie posiadała szereg wymaganej przeze mnie funkcjonalności, jak na przykład liczniki, abym mógł w trakcie treningu odliczać wykonane serie, alarmów, które odliczałyby przerwy pomiędzy seriami, przycisków informacyjnych otwierających okno dialogowe z informacjami dotyczącymi na przykład wymienianego miejsca, czy edycji styli tekstu za pomocą znaczników, zamiast przycisków, czy odrębnej klawiatury. Aplikacja nie powinna wymagać ode mnie dużego wysiłku do tworzenia zapisków pracując nad implementacją, ucząc się nowych rzeczy, zapisując myśli w nocy zaraz przed spaniem, oraz nie powinna wymagać długiego czasu oczekiwania na uruchomienie i synchronizację danych. Chciałem, aby aplikacja była szybka, prosta i przejrzysta jednocześnie pozwalając użytkownikowi na dodawania, edycję oraz interakcję różnego typu widgetów, alarmów czy powiadomień, aby aplikacja mogła również służyć w stanie, gdy aktualnie nie jest edytowana. W moich intencjach było, aby interfejs użytkownika był dopasowany do moich potrzeb i gustu, aby aplikacja odpowiadała mi pod wieloma względami nie tylko w kwestii oferowanych funkcji, ale również wyglądu, czy rozmieszczeniem i dostępnością funkcji, przycisków, czy widgetów tak, aby te najczęściej używane były najłatwiej dostępne i proste w obsłudze.

\section{Opis Aplikacji}

Aplikacja MobiNote służy do zarządzania notatkami w sposób opisany powyżej.
Głównym zamysłem aplikacji jest pomoc korzystającemu w prowadzeniu, tworzeniu i zapisywaniu notatek, w których skład wchodzi nie tylko tekst, ale również przyciski, obrazy, listy, liczniki i różnego rodzaju pomocne widgety.
Aplikacja umożliwia tworzenie i utrzymywanie brudnopisu używanego podczas codziennych czynności, jak również przejrzystych, dopracowanych i przystępnych  notatek.

Wytworzone notatki mogą być używane w formie brudnopisu, między innymi podczas: treningu na siłowni, organizacji przyjęcia, nauki z wykorzystaniem sesji pomodoro, tworzeniu listy zakupów oraz wielu innych codziennych czynności.

Mogą także w przyjemny dla oka sposób przechowywać i wyświetlać informacje przygotowane w celu tworzenia dziennika, notatek do nauki, spisu pomysłów i ważnych myśli, czy tworzenia różnego rodzaju list i opisów, takich jak lista miejsc, które użytkownik chciałby odwiedzić wraz z opisem i zdjęciami miejsc, które chciałby tam zobaczyć.

Interaktywność notatki jest zapewniana poprzez możliwość tworzenia zawartości na bierząco za pomocą klawiatury i dostępnego interfejsu użytkownika.
Uzytkownik może tworzyć styl tekstu za pomocą znaczników dodawanych wewnątrz tekstu w odpowiednich miejscach, podobnie do języka znaczników Markdown.
Dodawając odpowiednie znaczniki w tekście można ustawić wielkość czcionki w danym paragrafie, kursywe, podkreślenie, przekreślenie, a także pogrubienie.
Tekst jest automatycznie dostosowywany wraz z dodaniem znaczników, co pozawala na bierząco obserwować i dostosowywać style i wielkości czcionki do potrzeb korzystającego.

Notatki będą zapisywane głównie lokalnie, co pozwoli zaoszczędzić czas ładowania, jak również pominąć problemy związane z synchronizacją.

Interfejs użytkownika jest prosty i przejrzysty, posiada motyw ciemny(\textbf{dark}), jasny(\textbf{light}) oraz ułatwiający użytkowanie(\textbf{easy}) dla osób widzących słabiej.

\chapter{Instalacja}

Kod żródłowy aplikacji wraz plikiem \textbf{.apk} można uzyskać pobierając repozytorium \textbf{MobiNote} pod linkiem:
\url{https://github.com/bsobocki/MobiNote}.

\subsubsection{Aplikacje z nieznanych źródeł}
Do zaintalowania aplikacji potrzbna jest możliwość instalowania nieznanych aplikacji za pomocą menadżera plików.
Aby ją uruchomić, należy:
\begin{compactitem}
    \item otworzyć \textbf{Ustawienia}
    \item w \textbf{Ustawieniach} przejść do sekcji \textbf{Aplikacje}
    \item następnie kliknąć na ikone menu (trzy kropki w prawym górnym rogu)
    \item przejść do opcji \textbf{Dostęp specjalny}
    \item następnie \textbf{Zainstaluj nieznane aplikacje}
    \item wybrać menadżera plików i włączyć dla niego tę opcję
\end{compactitem}

\textbf{Ważne!} Dla bezpieczeństwa po zainstalowaniu aplikacji warto ponownie wyłączyć możliwość instalowania aplikacji z nienzanych źródeł, jeśli więcej aplikacji nie będzie w ten sposób instalowane.

\subsubsection{Instalacja z pliku apk-release.apk}

Aby zainstalować aplikację na urządzeniu mobilnym z systemem Android:

\begin{compactitem}
    \setlength\itemsep{0mm}
    \item pobirać repozytorium za pomocą polecenia
    \newline
    \verb|git clone git@github.com:bsobocki/MobiNote.git|
    \item podłączyć urządzenie i przenieść pliki \textbf{apk-release.apk} do wybranego przez siebie miejsca docelowego na urządzeniu
    \item włączyć możliwość instalacji nienzanych aplikacji (szczegóły powyżej)
    \item w menagerze plików znaleźć miejsce docelowe pliku \textbf{apk-release.apk} i uruchomić
    \item kliknąć \textbf{Zaintaluj}
\end{compactitem}

\chapter{Podręcznik użytkownika}
\label{ch:manual}

\section{Strona główna}

Po uruchomieniu aplikacji użytkownik zostaje przeniesiony na stronę główną. Znajdzie tam:
\begin{compactitem}
    \item swoje notatki i zeszyty;
    \item przycisk w prawym dolnym rogu służący do tworzenia nowej notatki;
    \item dodatkowe elementy w menu górnego paska.
\end{compactitem}
Elementy w menu górnego paska umożliwiają zmianę motywu, czy reset bazy danych w celu usunięcia wszystkich notatek. 

\begin{figure}[ht]
    \centering
    \includegraphics[height=9cm]{images/strona_domowa.png}
    \quad\quad
    \includegraphics[height=9cm]{images/strona_domowa_opcje.png}
    \caption{Strona główna aplikacji MobiNote z wybranym motywem \textbf{dark}.}
\end{figure}

\subsection{Opcja \textit{Choose Theme}}

Po uruchomieniu tej opcji wyświetli się okienko dialogowe z wyborem motywu, jakiego użytkownik chciałby użyć. Do wyboru mamy trzy mowtywy: \textbf{dark}, \textbf{light} oraz \textbf{easy}.

Motywy \textbf{dark} oraz \textbf{light} służą jako główne motywy aplikacji, zachowując te same wielkości elementów tekstu i całej aplikacji.
Dla osób mających problemy z widocznością tekstu przy domyślnych ustawieniach wielkości i kolorów aplikacji przygotowany został motyw \textbf{easy}.
Motyw ten cechują kontrastujące ze sobą kolory, jak również zwiększone rozmiary czcionek, ikon, przycisków i paska narzędzi.

\begin{figure}[ht]
    \centering
    \includegraphics[height=8.5cm]{images/strona_domowa_motywy.png}
    \caption{Porównanie motywów na tej samej stronie głównej.}
    \label{fig:mainPage}
\end{figure}

\subsection{Opcja \textit{Rebuild Database}}

Opcja ta otwiera okno dialogowe z pytaniem o to, czy na pewno chcemy wykonać operację przebudowy bazy danych. Po zatwierdzeniu baza danych jest usuwana i budowana ponownie.

\subsection{Zeszyty}

Pod etykietą \textbf{\textit{Notebooks}} dostępne są dwa przyciski: \textbf{Trainings} oraz \textbf{Diet}. Są to roboczo dodane przyciski, które są przygotowane do rozszerzenia aplikacji o możliwość układania notatek w zeszyty, dla lepszej organizacji, a także wprowadzać etykiety.
Pomysł ten zostanie omówiony w \ref{ch:rozwoj} rozdziale.

\subsection{Notatki}

Kolejnym elementem strony głównej jest lista notatek znajdująca się pod etykietą \textbf{\textit{Recent Notes}}. Każdy element zawiera tytuł oraz pierwsze cztery surowe linie tekstu (zawierające znaki specjalne styli w przypadku akapitu będącego tekstem, lub string w formacie JSON reprezentujący widget zawarty w danym akapitie).
W prawym górnym rogu danych elementów znajduje się przycisk \textbf{X} służący do usunięcia z bazy danych notatki reprezentowanej przez ten element.
Po naciśnięciu na jeden z omawianych elementów użytkownik zostaje przeniesiony do ekranu wyświetlania i edycji wybranej notatki.

W prawym dolnym rogu ekranu widnieje okrągły przycisk z ikoną ołówka (Rysunek \ref{fig:mainPage}).
Po jego naciśnięciu korzystający zostaje przeniesiony do ekranu edycji, gdzie może stworzyć i zapisać nową notatkę.

\section{Edycja notatki}

Po wybraniu notatki, lub przejściu do tworzenia nowej, na ekranie pojawi się strona edycji notatki. Składa się ona z głównego paska strony, paska narzędzi oraz edytora.

\subsection{Pasek strony}

\subsubsection{Przycisk zapisu i powrotu}

Aby wyjść i zapisać notatkę korzystający używa przycisku powrotu.
Ważne jest, aby zamiast systemowych przycisków nawigacyjnych użyć właśnie tego przycisku nawigacyjnego dostępnego w górnym pasku, ponieważ wraz z powrotem do strony głównej zapisuje on notatkę, jeśli ta uległa zmianie.

Za zmianę notatki uznawane są:

\begin{compactitem}
    \item edycja tytułu;
    \item edycja tekstu w akapitie tekstowym;
    \item edycja widgetu w akapitie widgetu.
\end{compactitem}
\vspace{5mm}

\textbf{WAŻNE!} Nowa notatka nie zostaje utworzona w momencie otwarcia okna edycji, a dopiero poprzez użycie przycisku powrotu pod warunkiem, że jej tytuł lub zawartość uległy zmianie. Oznacza to, że przejście do edycji nowej notatki, a następnie powrót, nie zapiszą pustej notatki, jednak dodanie i usunięcie znaku umożliwią zapis przy powrocie.

\subsubsection{Edytor tytułu}

Jest to pole tekstowe widniejące zaraz obok ikony przycisku (zapis i powrót). Służy ono do edycji tytułu notatki.

\subsubsection{Przycisk zmiany opcji zapisu}

Ostatnim elementem paska jest \textbf{przycisk zmiany opcji zapisu}. Jest to przycisk typu \textbf{switch} domyślnie ustawiony na true. Każdorazowe kliknięcie zmienia jego logiczną wartość.
Wartość \textbf{true} oznacza zapis notatki w przypadku jej edycji, natomiast \textbf{false} oznacza brak zapisu stanu notatki, nawet jeśli została ona zmieniona.

\subsection{Pasek narzędzi}

W tym pasku znajdują się przyciski oznaczone ikonami.
W aktualnej wersji aplikacji dostępne są dwa przyciski: dodanie obrazu (ikona obrazu), oraz dodanie listy (ikona listy).

\subsubsection{Dodanie obrazu}

Do zawartości notatki można dodawać również obrazy. Aby to zrobić należy kliknąć przycisk z ikoną obrazu na pasku narzędzi edytora notatki, a następnie wybrać z urządzenia zdjęcie, które ma zostać wstawione. Zdjęcie zostanie dodane do notatki.

\subsubsection{Dodanie listy}

Po wybraniu tej opcji na ekranie pojawi się lista złożona początkowo z jednego elementu.
Jest on domyślnie ustawiony na typ \textbf{checkbox}. Oznacza to, że jest to wiersz zawierający checkbox oraz puste pole tekstowe.

\subsection{Edytor strony}

Edytor strony zajmuje resztę powierzchni ekranu. Jest polem, w którym następuje edycja tekstu oraz widgetów. Aktualny stan wizualny jest odświeżany na bieżąco, dzięki czemu użytkownik obserwuje zmiany w efekcie końcowym w czasie rzeczywistym.



\section{Edycja zawartości notatki}

Dla lepszego zrozumienia działania aplikacji MobiNote będą używane sformułowania \textbf{akapit tekstowy} oraz \textbf{akapit widgetów}.
Są to nazwy opisujące abstrakcję użytą podczas konstruowania struktury aplikacji.

\subsection{Akapity}

Notatki w aplikacji składają się z akapitów, które dzielimy na akapity tekstowe i akapity widgetów.
Każdy akapit może przechodzić w jeden z dostępnych trybów.
Obecnie wyróżnione są tryby:

\begin{compactitem}
    \item widoku;
    \item edycji;
    \item zaznaczenia;
    \item niewidoczny.
\end{compactitem}

\subsubsection{tryb widoku}

Akapit istnieje w trybie widoku, kiedy nie jest aktywny.
Akapit przechodzi w tryb widoku, w momencie przeniesienia aktywności na inny akapit (na przykład poprzez kliknięcie).

\subsubsection{tryb edycji}

Akapit przechodzi w tryb edycji, w momencie przeniesienia na niego aktywności.
Aktywność ta może zostać przeniesiona poprzez kliknięcie na elementy akapitu, lub w przypadku, gdy jest to akapit widgetów, a jego następnikiem jest akapit tekstowy, którego edytujący stara się usunąć.

\subsubsection{tryb zaznaczenia}

Akapit widgetów przechodzi w tryb zaznaczenia poprzez przytrzymanie jego elementów niebędących polem tekstowym (pole tekstowe ma już zarezerwowany ten gest na zaznaczanie tekstu).

\subsubsection{tryb niewidoczny}

Akapit przestaje być widoczny w momencie przejścia edytora w kierunku pionowym wystarczająco do wysunięcia akapitu poza edytor.

\pagebreak

\subsection{Edycja akapitów tekstowych}

Każdy tekst zamykany jest w akapicie tekstowym od początku, aż do znaku końca linii. Oznacza to, że styl, jaki zostanie nadany na początku linii zostanie zaaplikowany na całą linię. Przykładem jest ustawienie akapitu jako nagłówka.

\subsubsection{Dodawanie}

Dodawanie akapitów tekstowych odbywa się za pomocą klawiatury. W momencie, gdy użytkownik podczas edycji doda znak nowej linii utworzy się nowy akapit będący następnikiem edytowanego.

\textbf{Ważne:} Podczas dodawania nowego akapitu widgetów utworzy się nowy akapit tekstowy będący jego następnikiem. Ma to na celu zachowanie możliwości ciągłej pracy z tekstem.

\subsubsection{Usuwanie}

Usunąć akapit tekstowy można poprzez naciśnięcie przycisku \textbf{delete} na samym początku tekstu. Jeśli poprzedni akapit jest akapitem tekstowym, wówczas pozostały tekst kopiowany jest na koniec poprzedniego akapitu.

\subsubsection{Ustawianie nagłówka}

Do formatowania tekstu używany jest język znaczników wzorowany na Markdown. Ustawienie nagłówka odbywa się poprzez wstawienie znaku \textbf{\#} na początku linii, przed tekstem.

\subsubsection{Dostępne nagłówki}
\begin{itemize}
    \setlength\itemsep{0mm}
    \item nagłówek1 \textbf{\#}
    \item nagłówek2 \textbf{\#\#}
    \item nagłówek3 \textbf{\#\#\#}
    \item nagłówek4 \textbf{\#\#\#\#}
\end{itemize}

Brak nagłówka oznacza, że dana linia jest zwykłym akapitem o domyślnej wielkości czcionki i odstępów oraz nie zawiera pogrubienia.

\pagebreak

\subsubsection{Odkrywanie znaków nagłówka}

Edytor posiada funkcję odkrywania znaków nagłówka, w celu ułatwienia edycji danego nagłówka. Dzięki temu użytkownik może zobaczyć który aktualnie nagłówek jest wybrany oraz w prosty sposób przechodzić pomiędzy typami nagłówków dodając bądź usuwając znak \textbf{\#}.

\begin{figure}[ht]
    \centering
    \includegraphics[height=6cm]{images/pokazywanie_naglowkow.png}
    \caption{Odkryte znaki w edytowanych nagłówkach.}
\end{figure}

\subsubsection{Ustawianie stylów}
\label{subsec:ustawianieStyli}

Kolejnym ważnym elementem aplikacji MobiNote są style tekstu. Użytkownik, chcąc sformatować tekst w odpowiednim stylu, wprowadza do niego symbole specjalne oznaczające konkretne style.
Odbywa się to według poniższych zasad:

\begin{itemize}
    \setlength\itemsep{2mm}

    \item za początek oznaczenia stylu uznawany jest wzór: \newline
    \verb|[dowolny znak][symbol stylu][znak niebędący białym znakiem]|
    
    \item za koniec oznaczenia stylu uznawany jest wzór: \newline
    \verb|[znak niebędący białym znakiem][symbol stylu]|

    \item dla każdego znacznika początkowego stylu musi istnieć znacznik końcowy;
    
    \item style mogą być łączone -- pomiędzy znacznikami stylu A można dodać znaczniki stylu B;

    \item style nie mogą się przecinać -- jeśli dodajemy znacznik początkowy stylu B pomiędzy znacznikami stylu A, to znacznik końca stylu B powinien wystąpić przed znacznikiem końca stylu A. 
\end{itemize}

\begin{figure}[ht]
    \centering
    \includegraphics[width=8cm]{images/style.png}
    \caption{Zagnieżdżone style w prawidłowy i nieprawidłowy sposób}
    \vspace{3mm}
\end{figure}

\begin{figure}[ht]
    \centering
    \includegraphics[width=8cm]{images/style_surowy_tekst.png}
    \caption{Surowy tekst ukazujący rozmieszczenie znaków.}
    \vspace{3mm}
\end{figure}

\subsubsection{Znaki specjalne}

Każdy dostępny styl tekstu oznaczony jest za pomocą znaków:

\begin{compactitem}
    
    \item [*] \hspace{1mm} -- pogrubienie
    \item [\^{}] \hspace{1mm} -- kursywa
    \item [\_{}] \hspace{1mm} -- podkreślenie
    \item [\~{}] \hspace{1mm} -- przekreślenie
\end{compactitem}

\subsubsection{Odkrywanie znaków specjalnych}

Edytor tekstu posiada funkcję odwijania stylu w momencie, gdy kursor znajduje się bezpośrednio w środku stylu. Odwijany jest tylko styl, którego jeden ze znaków specjalnych znajduje się najbliżej kursora. Funkcjonalność ta została wprowadzona, aby użytkownik mógł łatwiej edytować i usuwać style.
Dzięki temu może dostrzec, gdzie konkretnie znajdują się symbole danego stylu w tekście.


\begin{figure}[ht]
    \centering
    \includegraphics[width=8cm]{images/pokazywanie_znakow_specjalnych.png}
    \caption{Przykłady odkrytych znaków specjalnych.}
    \vspace{3mm}
\end{figure}

\subsubsection{Usuwanie i edycja stylów}

Aby usunąć dany styl wystarczy usunąć reprezentujące go znaki specjalne. Poza odkrytymi znakami reszta jest ukryta, jednak wszystkie znaki nadal występują w tekście. Edytujący może zatem kliknąć bezpośrednio przed widoczny początkowy symbol lub poza tekst oznaczony danym stylem i naciskając przycisk \textbf{delete}, na klawiaturze urządzenia, usunąć dany niewidoczny symbol zakończenia stylu.

\subsection{Edycja akapitów widgetów}

Elementami tych akapitów są widgety. W aktualnej wersji aplikacji dostępne są dwa widgety, które mogą być bezpośrednimi elementami akapitów widgetów.
Są to \textbf{obrazy} i \textbf{listy}.

\subsubsection{Dodawanie}
\label{sub:dodawanieWidgetow}

Dodawanie akapitów widgetów odbywa się poprzez użycie przycisków z paska narzędzi edytora notatki.

\subsubsection{Usuwanie}
\label{sub:usuwanieWidgetow}

Akapit widgetów można usunąć poprzez usunięcie jego wszystkich głównych elementów. W obecnej wersji możliwe jest posiadanie tylko jednego głównego elementu, jednak w planie rozwoju aplikacji przewidziane jest, aby akapity mogły przechowywać i używać większej ilości widgetów, w zależności od rodzajów używanych widgetów.

\subsubsection{Tryb Edycji}

Przejście do trybu edycji odbywa się poprzez interakcję z widgetem, np naciśnięcie na obraz, bądź edycja tekstu w elemencie listy.
W przypadku obrazów tryb edycji jest oznaczony poprzez dodanie obramowania do zdjęcia, jak również ikony w dolnej części obramowania służącej do zmiany rozmiaru zdjęcia.

\begin{figure}[ht]
    \centering
    \includegraphics[width=6cm]{images/tryb_edycji.png}
    \caption{Tryb edycji przykładowych widgetów głównych.}
    \label{fig:trybEdycji}
    \vspace{3mm}
\end{figure}

\subsubsection{Tryb zaznaczenia}

Przejście w tryb zaznaczenia odbywa się poprzez naciśnięcie i przytrzymanie konkretnych części widgetów. W przypadku obrazu jest to sam obraz, z kolei w przypadku list są to etykiety przypięte do pola tekstowego (w zależności od wyboru mogą być to: checkbox, etykiety tekstowe czy liczniki).

\begin{figure}[ht]
    \centering
    \includegraphics[width=6cm]{images/tryb_zaznaczenia.png}
    \caption{Tryb zaznaczenia przykładowych widgetów głównych.}
    \label{fig:trybZaznaczenia}
    \vspace{3mm}
\end{figure}


\subsection{Obrazy}

Jednym z głównych widgetów są obrazy dodawane poprzez użycie przycisku na pasku narzędzi. Obraz wybierany jest z pamięci urządzenia.

\subsubsection{Edycja}

Aby zmienić rozmiar zdjęcia użytkownik musi przejść w tryb edycji poprzez kliknięcie na zdjęcie, a następnie przeciągać ikonę ze strzałką (Rysunek \ref{fig:trybEdycji}). Automatycznie dostosowywany będzie rozmiar obrazu.

\subsubsection{Usunięcie i zmiana}

Po wejściu w tryb zaznaczenia ukażą się dwie ikony: ikona usunięcia obrazu oraz ikona zmiany obrazu (Rysunek \ref{fig:trybZaznaczenia}).
Wybór pierwszej ikony skutkuje usunięciem obrazu wraz z akapitem, natomiast drugiej -- przejściem do pamięci urządzenia w celu wyboru zdjęcia.

\subsection{Listy}

Dostępne są różne rodzaje list. Różnią się one głównie etykietą oraz niewielkimi elementami, jak na przykład dostępne przekreślenie i zmiana koloru tekstu przy odznaczonej pozycji.

\subsubsection{Edycja}

Edycja list następuje poprzez interakcję użytkownika. Użytkownik może naciskać bezpośrednio na etykiety przy polu tekstowym w celu ich edycji (checkbox, licznik), jak również edytować tekst wiersza.

W przypadku niektórych typów wiersza możliwe jest jego odznaczenie. Wówczas tekst wiersza zmienia kolor na szary, a sam tekst zostaje przekreślony.

\subsubsection{Dodawanie wierszy}

Dodawanie wierszy następuje na dwa sposoby:

\begin{enumerate}
    \item W dowolnym miejscu listy poprzez dodanie znaku nowej linii.
    \item Na koniec listy poprzez kliknięcie przycisku "+" pod etykietami w trybie zaznaczenia. 
\end{enumerate}

Dodawanie wierszy poprzez znak nowej linii przenosi tekst na prawo od miejsca dodania znaku nowej linii do nowego wiersza.

\subsubsection{Usuwanie wierszy}

Usuwanie wierszy jest możliwe w trybie zaznaczania poprzez klikanie na etykiety ze znakiem "x".

\subsubsection{Zmiana typu wierszy}

Użytkownik może zmienić typ wszystkich wierszy w danej liście poprzez przejście w tryb zaznaczenia, a następnie w górym pasku nad listą (Rysunek \ref{fig:trybZaznaczenia}) wybrać jedną z dostępnych opcji.

\subsubsection{Dostępne typy wiersza}

\begin{itemize}
    \item checkbox;
    \item numerowane;
    \item oznaczone przez symbol "\--{}";
    \item naznaczane przez symbol "*" (w nowszej wersji aplikacji spodziewane jest tworzenie etykiet przez użytkownika);
    \item licznik.
\end{itemize}

Aktualnie listy są jednopoziomowe, bez możliwości zmiany wysunięcia od początku wiersza. Możliwość ta jest przewidziana w dalszym rozwoju aplikacji.

\subsubsection{Licznik}

Licznik służy do odliczania rzeczy opisywanej przez tekst wiersza. Z każdym kliknięciem zwiększana jest wartość licznika (lewa część etykiety), aż do osiągnięcia celu licznika (liczba w prawej części licznika). Wraz z osiągnięciem ustalonego celu wiersz zostaje odznaczony.

Możliwa jest zmiana wartości celu licznika. Aby to zrobić wystarczy nacisnąć na liczbę w prawej części licznika, a następnie wybrać liczbę na klawiaturze i zatwierdzić. Po zmianie następuje reset licznika.


\begin{figure}[ht]
    \centering
    \includegraphics[width=6cm]{images/liczniki.png}
    \caption{Edycja celu jednego z liczników.}
    \vspace{3mm}
\end{figure}

\chapter{Implementacja}
    
Implementując aplikację MobiNote starałem się zachowywać czysty kod, podzielony na jak najmniejsze części pod względem odpowiedzialności i zadań klas oraz metod, jak również czyste repozytorium zachowując odpowiedni podział i strukturę katalogów, modułów i pozostałych plików. Przykładałem dużą wagę do jakości oraz wykonania, aby powrót do poszczególnych warstw i miejsc w aplikacji był prosty, a sama jej struktura była przejrzysta i zrozumiała. 

\section{Technologie}

\subsection{flutter}

Aplikacja została napisana przy użyciu frameworka flutter w języku Dart.

Wybierając technologie, kierowałem się kryteriami takimi jak: prostota, wieloplatformowość, estetyka bez dużego wkładu w kreowanie komponentów na własną rękę oraz wydajność. Jedną z najbardziej polecanych framework'ów na stronie \href{https://www.linkedin.com/pulse/best-9-mobile-app-development-frameworks-2023-sstech-system/}{linkedin.com} był \textbf{flutter}. Jak podaje strona główna \href{https://www.flutter.dev}{flutter.dev}:

"Flutter is an open source framework by Google for building beautiful, natively compiled, multi-platform applications from a single codebase."

Składnia języka Dart jest stosunkowo prosta i przejrzysta, natomiast framework flutter pozwala na programowanie aplikacji mobilnych w prosty sposób na platformy IOS oraz Android o dużej wydajności zważywszy na to, że język Dart jest kompilowany do natywnego kodu. Pozwala to na tworzenie aplikacji szybko i bez wielkiego nakładu pracy, dając przy tym duże możliwości i przyjemny dla oka rozbudowany interfejs użytkownika.

%https://itcraftapps.com/pl/blog/flutter-w-swiecie-aplikacji-mobilnych-czy-to-przyszlosc-programowania/ 

\subsection{SQLite}

Do przechowywania notatek została wykorzystana baza danych SQLite. Aplikacja wykorzystuje bibliotekę \textbf{sqlite3}, umożliwiającą wykonywanie operacji na lokalnej bazie danych SQLite. Pozwala na ona na zapis, odczyt a także manipulację danymi. Dodatkowo wykorzystywany jest pakiet \href{https://pub.dev/packages/drift}{drift}, który jest rodzajem systemu ORM umożliwiającym mapowanie między obiektami języka Dart oraz tabelami bazy danych SQLite. Pozwala to na pracę bezpośrednio w kodzie Dart używając dostępnej funkcjonalności bez konieczności pisania zapytań bezpośrednio w języku SQL.

\section{Organizacja repozytorium}

Kod aplikacji został podzielony na moduły i zorganizowany w zależności od poziomu abstrakcji i zastosowań.

Główny katalog z kodem źródłowym \textbf{lib} zawiera:

\begin{itemize}
    \item plik \textbf{main.dart} z wywołaniem głównej funkcji main budującej aplikację;
    \item katalog \textbf{database};
    \item katalog \textbf{logic};
    \item katalog \textbf{screens}.
\end{itemize}

\subsubsection{database}

Zawiera definicję bazy danych, struktury oraz metody z nią związane. Posiada wygenerowany na podstawie pliku \textbf{database\_{}def.dart} plik \textbf{database\_{}def.g.dart} zawierający struktury odzwierciedlające tabele bazy danych w klasy języka Dart. 

\subsubsection{logic}

Znajdują się tam definicje struktur reprezentujących stan komponentów, definicje typów (stylów tekstu, widgetów, znaczników itd.), mapowanie kluczy (String) na style tekstu, znaczniki i elementy, logika parsera, funkcje pomocnicze wraz z funkcjonalnością aplikacji niebędącej bezpośrednią częścią widgetów.

\subsubsection{screens}

Dostępne są tam definicje stron wraz z funkcjonalnością i metodami ich komponentów, jak również definicje motywów i kolorów aplikacji.

\section{Wykorzystane rozwiązania}

\subsubsection{Programowanie Obiektowe}

Dart promuje programowanie obiektowe, dlatego też w projekcie został zastosowany paradygmat obiektowego programowania wraz z jego mechanizmami. Przykładami tych mechanizmów może być wielokrotne zastosowanie abstrakcji różnych poziomów, dziedziczenie i polimorfizm.

Przykładem zastosowania wszystkich powyższych mechanizmów jest reprezentacja, tworzenie oraz zarządzanie widgetami. 
Dzięki zastosowaniu opisanych mechanizmów możliwe było zaimplementowanie fabryki, listy, oraz akapitów trzymanych w edytorze.

\subsubsection{Wzorce projektowe}

W powyższych opisach możemy zauważyć użycie wzorców projektowych spotykanych podczas pisania aplikacji mobilnej. Niektóre z nich opisane są na stronie \href{https://flutterdesignpatterns.com/}{flutterdesignpatterns.com}. Są to między innymi:

\begin{itemize}
    \item \textbf{Fabryka} -- użyty w przypadku fabryki widgetów \textit{NoteEditorWidgetFactory};
    \item \textbf{Singleton} -- używany w przypadku dostępu do bazy danych oraz dostępu do ustawionego motywu aplikacji (\textit{MobiNoteDatabase}, \textit{MobiNoteTheme});
    \item \textbf{Kompozyt} -- używany w przypadku struktur drzewiastych takich jak na przykład obiekty \textit{SpanTree}, czy \textit{NoteWidgetData};
\end{itemize}

\subsubsection{Testy Jednostkowe}

Część logiki niezwiązanej bezpośrednio z wyświetlaniem i organizacją komponentów i ich wyglądu testowałem za pomocą testów jednostkowych od razu podczas ich implementacji. Dzięki temu uniknąłem długich godzin poszukiwania problemów, gdy któryś z komponentów mógł nie działać poprawnie. Przyczyną niepoprawnego działanie może być nie tylko algorytm, ale i sam widget i funkcjonowanie frameworka flutter. Dostosowując szczegółowość testów jednostkowych starałem się pokryć przypadki użycia jeszcze przed użyciem danych klas, metod i algorytmów w aplikacji. Przetestowany został w ten sposób parser języka znaczników stylów, wraz z pełną funkcjonalnością, jak również konwersja reprezentacji widgetów do formatu JSON.

\subsection{Baza danych}

Do zarządzania dostępem i manipulacją bazy danych tworzony jest jeden obiektu typu \textbf{MobiNoteDatabase} dla całego projektu. Jest on tworzony i udostępniany za pomocą biblioteki do zarządzania stanem \textbf{GetX}. 

W implementacji aplikacji MobiNote, GetX pozwala na tworzenie instancji zarządzającej dostępem do bazy danych raz i udostępnianie jej w różnych miejscach aplikacji. Struktura aplikacji wymaga tylko jednej instancji obiektu MobiNoteDatabase, dlatego tworzone jest jedno połączenie w jej konstruktorze, a sam obiekt przechowywany jest z pomocą GetX

\begin{verbatim}
    Get.put(MobiNoteDatabase());
\end{verbatim}

\noindent Następnie za pomocą funkcji \textbf{find} może zostać udostępniany

\begin{verbatim}
    final database = Get.find<MobiNoteDatabase>();
\end{verbatim}

Umożliwia to zwiększenie wydajności, ponieważ unikamy otwierania połączenia z bazą danych za każdym razem, gdy jest ono potrzebne (obie strony: strona główna, oraz edytor notatki używają połączenia z bazą danych do swoich funkcjonalności). Używany zatem jest tutaj wzorzec Singleton. Z racji tego, że nie ma potrzeby otwierania równoległych połączeń (używana jest tylko jedna ze stron i ich funkcjonalność naraz) wzorzec ten został zastosowany w implementacji.

\subsection{Dobre praktyki}

\subsubsection{Leniwe tworzenie widgetów w ListView}

Podczas implementacji listy notatek w stronie głównej oraz edytora zawartości notatki użyty został widget ListView poprzez wywołanie twz. \textit{named constructor} \textbf{ListView.builder}. Praktyka ta ma dobre zastosowanie przy długich listach, ponieważ używając konstruktora \textbf{builder} widgety zawarte w ListView budowane są w sposób leniwy, na bieżąco, jeśli istnieje potrzeba ich wyświetlenia. Z racji tego, że budowanie jest szybkie i nieskomplikowane pozwala to uniknąć przechowywania dużej ilości widgetów (znajduje to pozytywne skutki przy długich, skomplikowanych notatkach).
%https://docs.flutter.dev/cookbook/lists/long-lists

\subsubsection{Reprezentacja widgetów}

Każdy widget reprezentowany jest poprzez obiekt typu pochodnej klasy WidgetData przechowywującej parametry danego widgetu potrzebne do jego prawidłowego wyświetlania oraz funkcjonowania. Oddzielana jest warstwa wyświetlania od warstwy informacji danego widgetu. Pozwala to na zapis stanu widgetu do bazy danych, ponieważ zawarta w obiektach reprezentujących warstwę informacji jest funkcja konwersji do formatu JSON, który jest przechowywany w bazie danych jako fragment zawartości danej notatki.

\subsubsection{ID Generator}

Widgety zawierają identyfikatory do odróżniania ich w rodzicu lub liście zwierającej je. Ma to zastosowanie m.in. w znajdowaniu miejsca w liście np do dodawania nowego elementu, czy usuwania i zmiany elementów. Lepiej nadać id i na podstawie znalezienia id na liście pobierać index, zamiast nadawać i aktualizować indeksy za każdym razem elementom przy manipulacji listy.

Do generowania identyfikatorów w odpowiednich miejscach w kodzie służy obiekt klasy \textbf{IdGenerator}.

\subsubsection{Fabryka widgetów}

Widgety tworzone są na podstawie obiektów reprezentacyjnych (opisanych powyżej) w klasie \textbf{NoteEditorWidgetFactory}. Na podstawie pola \textbf{\textit{type}} wybierany jest typ widgetu, który następnie tworzony jest z parametrów obiektu reprezentującego. Używa klasy IdGenerator do generowania identyfikatorów, przez co obsługa przydzielania id jest robiona w jednym miejscu, w fabryce, która jest przekazywana dzieciom i elementom danych widgetów. Pozwala to zachować spójność w aktualnej implementacji, jak również przy dalszym rozwoju.

%%%%% BIBLIOGRAFIA

%\begin{thebibliography}{1}
%\bibitem{example} \ldots
%\end{thebibliography}

\end{document}