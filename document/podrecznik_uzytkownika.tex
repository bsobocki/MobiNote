\chapter{Podręcznik użytkownika}
\label{ch:manual}

\section{Strona główna}

Po uruchomieniu aplikacji użytkownik zostaje przeniesiony na stronę główną. Znajdzie tam:
\begin{compactitem}
    \item swoje notatki i zeszyty;
    \item przycisk w prawym dolnym rogu służący do tworzenia nowej notatki;
    \item dodatkowe elementy w menu górnego paska.
\end{compactitem}
Elementy w menu górnego paska umożliwiają zmianę motywu, czy reset bazy danych w celu usunięcia wszystkich notatek. 

\begin{figure}[ht]
    \centering
    \includegraphics[height=9cm]{images/strona_domowa.png}
    \quad\quad
    \includegraphics[height=9cm]{images/strona_domowa_opcje.png}
    \caption{Strona główna aplikacji MobiNote z wybranym motywem \textbf{dark}.}
\end{figure}

\subsection{Opcja \textit{Choose Theme}}

Po uruchomieniu tej opcji wyświetli się okienko dialogowe z wyborem motywu, jakiego użytkownik chciałby użyć. Do wyboru mamy trzy mowtywy: \textbf{dark}, \textbf{light} oraz \textbf{easy}.

Motywy \textbf{dark} oraz \textbf{light} służą jako główne motywy aplikacji, zachowując te same wielkości elementów tekstu i całej aplikacji.
Dla osób mających problemy z widocznością tekstu przy domyślnych ustawieniach wielkości i kolorów aplikacji przygotowany został motyw \textbf{easy}.
Motyw ten cechują kontrastujące ze sobą kolory, jak również zwiększone rozmiary czcionek, ikon, przycisków i paska narzędzi.

\begin{figure}[ht]
    \centering
    \includegraphics[height=8.5cm]{images/strona_domowa_motywy.png}
    \caption{Porównanie motywów na tej samej stronie głównej.}
    \label{fig:mainPage}
\end{figure}

\subsection{Opcja \textit{Rebuild Database}}

Opcja ta otwiera okno dialogowe z pytaniem o to, czy na pewno chcemy wykonać operację przebudowy bazy danych. Po zatwierdzeniu baza danych jest usuwana i budowana ponownie.

\subsection{Zeszyty}

Pod etykietą \textbf{\textit{Notebooks}} dostępne są dwa przyciski: \textbf{Trainings} oraz \textbf{Diet}. Są to roboczo dodane przyciski, które są przygotowane do rozszerzenia aplikacji o możliwość układania notatek w zeszyty, dla lepszej organizacji, a także wprowadzać etykiety.
Pomysł ten zostanie omówiony w \ref{ch:rozwoj} rozdziale.

\subsection{Notatki}

Kolejnym elementem strony głównej jest lista notatek znajdująca się pod etykietą \textbf{\textit{Recent Notes}}. Każdy element zawiera tytuł oraz pierwsze cztery surowe linie tekstu (zawierające znaki specjalne styli w przypadku akapitu będącego tekstem, lub string w formacie JSON reprezentujący widget zawarty w danym akapitie).
W prawym górnym rogu danych elementów znajduje się przycisk \textbf{X} służący do usunięcia z bazy danych notatki reprezentowanej przez ten element.
Po naciśnięciu na jeden z omawianych elementów użytkownik zostaje przeniesiony do ekranu wyświetlania i edycji wybranej notatki.

W prawym dolnym rogu ekranu widnieje okrągły przycisk z ikoną ołówka (Rysunek \ref{fig:mainPage}).
Po jego naciśnięciu korzystający zostaje przeniesiony do ekranu edycji, gdzie może stworzyć i zapisać nową notatkę.

\section{Edycja notatki}

Po wybraniu notatki, lub przejściu do tworzenia nowej, na ekranie pojawi się strona edycji notatki. Składa się ona z głównego paska strony, paska narzędzi oraz edytora.

\subsection{Pasek strony}

\subsubsection{Przycisk zapisu i powrotu}

Aby wyjść i zapisać notatkę korzystający używa przycisku powrotu.
Ważne jest, aby zamiast systemowych przycisków nawigacyjnych użyć właśnie tego przycisku nawigacyjnego dostępnego w górnym pasku, ponieważ wraz z powrotem do strony głównej zapisuje on notatkę, jeśli ta uległa zmianie.

Za zmianę notatki uznawane są:

\begin{compactitem}
    \item edycja tytułu;
    \item edycja tekstu w akapitie tekstowym;
    \item edycja widgetu w akapitie widgetu.
\end{compactitem}
\vspace{5mm}

\textbf{WAŻNE!} Nowa notatka nie zostaje utworzona w momencie otwarcia okna edycji, a dopiero poprzez użycie przycisku powrotu pod warunkiem, że jej tytuł lub zawartość uległy zmianie. Oznacza to, że przejście do edycji nowej notatki, a następnie powrót, nie zapiszą pustej notatki, jednak dodanie i usunięcie znaku umożliwią zapis przy powrocie.

\subsubsection{Edytor tytułu}

Jest to pole tekstowe widniejące zaraz obok ikony przycisku (zapis i powrót). Służy ono do edycji tytułu notatki.

\subsubsection{Przycisk zmiany opcji zapisu}

Ostatnim elementem paska jest \textbf{przycisk zmiany opcji zapisu}. Jest to przycisk typu \textbf{switch} domyślnie ustawiony na true. Każdorazowe kliknięcie zmienia jego logiczną wartość.
Wartość \textbf{true} oznacza zapis notatki w przypadku jej edycji, natomiast \textbf{false} oznacza brak zapisu stanu notatki, nawet jeśli została ona zmieniona.

\subsection{Pasek narzędzi}

W tym pasku znajdują się przyciski oznaczone ikonami.
W aktualnej wersji aplikacji dostępne są dwa przyciski: dodanie obrazu (ikona obrazu), oraz dodanie listy (ikona listy).

\subsubsection{Dodanie obrazu}

Do zawartości notatki można dodawać również obrazy. Aby to zrobić należy kliknąć przycisk z ikoną obrazu na pasku narzędzi edytora notatki, a następnie wybrać z urządzenia zdjęcie, które ma zostać wstawione. Zdjęcie zostanie dodane do notatki.

\subsubsection{Dodanie listy}

Po wybraniu tej opcji na ekranie pojawi się lista złożona początkowo z jednego elementu.
Jest on domyślnie ustawiony na typ \textbf{checkbox}. Oznacza to, że jest to wiersz zawierający checkbox oraz puste pole tekstowe.

\subsection{Edytor strony}

Edytor strony zajmuje resztę powierzchni ekranu. Jest polem, w którym następuje edycja tekstu oraz widgetów. Aktualny stan wizualny jest odświeżany na bieżąco, dzięki czemu użytkownik obserwuje zmiany w efekcie końcowym w czasie rzeczywistym.



\section{Edycja zawartości notatki}

Dla lepszego zrozumienia działania aplikacji MobiNote będą używane sformułowania \textbf{akapit tekstowy} oraz \textbf{akapit widgetów}.
Są to nazwy opisujące abstrakcję użytą podczas konstruowania struktury aplikacji.

\subsection{Akapity}

Notatki w aplikacji składają się z akapitów, które dzielimy na akapity tekstowe i akapity widgetów.
Każdy akapit może przechodzić w jeden z dostępnych trybów.
Obecnie wyróżnione są tryby:

\begin{compactitem}
    \item widoku;
    \item edycji;
    \item zaznaczenia;
    \item niewidoczny.
\end{compactitem}

\subsubsection{tryb widoku}

Akapit istnieje w trybie widoku, kiedy nie jest aktywny.
Akapit przechodzi w tryb widoku, w momencie przeniesienia aktywności na inny akapit (na przykład poprzez kliknięcie).

\subsubsection{tryb edycji}

Akapit przechodzi w tryb edycji, w momencie przeniesienia na niego aktywności.
Aktywność ta może zostać przeniesiona poprzez kliknięcie na elementy akapitu, lub w przypadku, gdy jest to akapit widgetów, a jego następnikiem jest akapit tekstowy, którego edytujący stara się usunąć.

\subsubsection{tryb zaznaczenia}

Akapit widgetów przechodzi w tryb zaznaczenia poprzez przytrzymanie jego elementów niebędących polem tekstowym (pole tekstowe ma już zarezerwowany ten gest na zaznaczanie tekstu).

\subsubsection{tryb niewidoczny}

Akapit przestaje być widoczny w momencie przejścia edytora w kierunku pionowym wystarczająco do wysunięcia akapitu poza edytor.

\pagebreak

\subsection{Edycja akapitów tekstowych}

Każdy tekst zamykany jest w akapicie tekstowym od początku, aż do znaku końca linii. Oznacza to, że styl, jaki zostanie nadany na początku linii zostanie zaaplikowany na całą linię. Przykładem jest ustawienie akapitu jako nagłówka.

\subsubsection{Dodawanie}

Dodawanie akapitów tekstowych odbywa się za pomocą klawiatury. W momencie, gdy użytkownik podczas edycji doda znak nowej linii utworzy się nowy akapit będący następnikiem edytowanego.

\textbf{Ważne:} Podczas dodawania nowego akapitu widgetów utworzy się nowy akapit tekstowy będący jego następnikiem. Ma to na celu zachowanie możliwości ciągłej pracy z tekstem.

\subsubsection{Usuwanie}

Usunąć akapit tekstowy można poprzez naciśnięcie przycisku \textbf{delete} na samym początku tekstu. Jeśli poprzedni akapit jest akapitem tekstowym, wówczas pozostały tekst kopiowany jest na koniec poprzedniego akapitu.

\subsubsection{Ustawianie nagłówka}

Do formatowania tekstu używany jest język znaczników wzorowany na Markdown. Ustawienie nagłówka odbywa się poprzez wstawienie znaku \textbf{\#} na początku linii, przed tekstem.

\subsubsection{Dostępne nagłówki}
\begin{itemize}
    \setlength\itemsep{0mm}
    \item nagłówek1 \textbf{\#}
    \item nagłówek2 \textbf{\#\#}
    \item nagłówek3 \textbf{\#\#\#}
    \item nagłówek4 \textbf{\#\#\#\#}
\end{itemize}

Brak nagłówka oznacza, że dana linia jest zwykłym akapitem o domyślnej wielkości czcionki i odstępów oraz nie zawiera pogrubienia.

\pagebreak

\subsubsection{Odkrywanie znaków nagłówka}

Edytor posiada funkcję odkrywania znaków nagłówka, w celu ułatwienia edycji danego nagłówka. Dzięki temu użytkownik może zobaczyć który aktualnie nagłówek jest wybrany oraz w prosty sposób przechodzić pomiędzy typami nagłówków dodając bądź usuwając znak \textbf{\#}.

\begin{figure}[ht]
    \centering
    \includegraphics[height=6cm]{images/pokazywanie_naglowkow.png}
    \caption{Odkryte znaki w edytowanych nagłówkach.}
\end{figure}

\subsubsection{Ustawianie stylów}
\label{subsec:ustawianieStyli}

Kolejnym ważnym elementem aplikacji MobiNote są style tekstu. Użytkownik, chcąc sformatować tekst w odpowiednim stylu, wprowadza do niego symbole specjalne oznaczające konkretne style.
Odbywa się to według poniższych zasad:

\begin{itemize}
    \setlength\itemsep{2mm}

    \item za początek oznaczenia stylu uznawany jest wzór: \newline
    \verb|[dowolny znak][symbol stylu][znak niebędący białym znakiem]|
    
    \item za koniec oznaczenia stylu uznawany jest wzór: \newline
    \verb|[znak niebędący białym znakiem][symbol stylu]|

    \item dla każdego znacznika początkowego stylu musi istnieć znacznik końcowy;
    
    \item style mogą być łączone -- pomiędzy znacznikami stylu A można dodać znaczniki stylu B;

    \item style nie mogą się przecinać -- jeśli dodajemy znacznik początkowy stylu B pomiędzy znacznikami stylu A, to znacznik końca stylu B powinien wystąpić przed znacznikiem końca stylu A. 
\end{itemize}

\begin{figure}[ht]
    \centering
    \includegraphics[width=8cm]{images/style.png}
    \caption{Zagnieżdżone style w prawidłowy i nieprawidłowy sposób}
    \vspace{3mm}
\end{figure}

\begin{figure}[ht]
    \centering
    \includegraphics[width=8cm]{images/style_surowy_tekst.png}
    \caption{Surowy tekst ukazujący rozmieszczenie znaków.}
    \vspace{3mm}
\end{figure}

\subsubsection{Znaki specjalne}

Każdy dostępny styl tekstu oznaczony jest za pomocą znaków:

\begin{compactitem}
    
    \item [*] \hspace{1mm} -- pogrubienie
    \item [\^{}] \hspace{1mm} -- kursywa
    \item [\_{}] \hspace{1mm} -- podkreślenie
    \item [\~{}] \hspace{1mm} -- przekreślenie
\end{compactitem}

\subsubsection{Odkrywanie znaków specjalnych}

Edytor tekstu posiada funkcję odwijania stylu w momencie, gdy kursor znajduje się bezpośrednio w środku stylu. Odwijany jest tylko styl, którego jeden ze znaków specjalnych znajduje się najbliżej kursora. Funkcjonalność ta została wprowadzona, aby użytkownik mógł łatwiej edytować i usuwać style.
Dzięki temu może dostrzec, gdzie konkretnie znajdują się symbole danego stylu w tekście.


\begin{figure}[ht]
    \centering
    \includegraphics[width=8cm]{images/pokazywanie_znakow_specjalnych.png}
    \caption{Przykłady odkrytych znaków specjalnych.}
    \vspace{3mm}
\end{figure}

\subsubsection{Usuwanie i edycja stylów}

Aby usunąć dany styl wystarczy usunąć reprezentujące go znaki specjalne. Poza odkrytymi znakami reszta jest ukryta, jednak wszystkie znaki nadal występują w tekście. Edytujący może zatem kliknąć bezpośrednio przed widoczny początkowy symbol lub poza tekst oznaczony danym stylem i naciskając przycisk \textbf{delete}, na klawiaturze urządzenia, usunąć dany niewidoczny symbol zakończenia stylu.

\subsection{Edycja akapitów widgetów}

Elementami tych akapitów są widgety. W aktualnej wersji aplikacji dostępne są dwa widgety, które mogą być bezpośrednimi elementami akapitów widgetów.
Są to \textbf{obrazy} i \textbf{listy}.

\subsubsection{Dodawanie}
\label{sub:dodawanieWidgetow}

Dodawanie akapitów widgetów odbywa się poprzez użycie przycisków z paska narzędzi edytora notatki.

\subsubsection{Usuwanie}
\label{sub:usuwanieWidgetow}

Akapit widgetów można usunąć poprzez usunięcie jego wszystkich głównych elementów. W obecnej wersji możliwe jest posiadanie tylko jednego głównego elementu, jednak w planie rozwoju aplikacji przewidziane jest, aby akapity mogły przechowywać i używać większej ilości widgetów, w zależności od rodzajów używanych widgetów.

\subsubsection{Tryb Edycji}

Przejście do trybu edycji odbywa się poprzez interakcję z widgetem, np naciśnięcie na obraz, bądź edycja tekstu w elemencie listy.
W przypadku obrazów tryb edycji jest oznaczony poprzez dodanie obramowania do zdjęcia, jak również ikony w dolnej części obramowania służącej do zmiany rozmiaru zdjęcia.

\begin{figure}[ht]
    \centering
    \includegraphics[width=6cm]{images/tryb_edycji.png}
    \caption{Tryb edycji przykładowych widgetów głównych.}
    \label{fig:trybEdycji}
    \vspace{3mm}
\end{figure}

\subsubsection{Tryb zaznaczenia}

Przejście w tryb zaznaczenia odbywa się poprzez naciśnięcie i przytrzymanie konkretnych części widgetów. W przypadku obrazu jest to sam obraz, z kolei w przypadku list są to etykiety przypięte do pola tekstowego (w zależności od wyboru mogą być to: checkbox, etykiety tekstowe czy liczniki).

\begin{figure}[ht]
    \centering
    \includegraphics[width=6cm]{images/tryb_zaznaczenia.png}
    \caption{Tryb zaznaczenia przykładowych widgetów głównych.}
    \label{fig:trybZaznaczenia}
    \vspace{3mm}
\end{figure}


\subsection{Obrazy}

Jednym z głównych widgetów są obrazy dodawane poprzez użycie przycisku na pasku narzędzi. Obraz wybierany jest z pamięci urządzenia.

\subsubsection{Edycja}

Aby zmienić rozmiar zdjęcia użytkownik musi przejść w tryb edycji poprzez kliknięcie na zdjęcie, a następnie przeciągać ikonę ze strzałką (Rysunek \ref{fig:trybEdycji}). Automatycznie dostosowywany będzie rozmiar obrazu.

\subsubsection{Usunięcie i zmiana}

Po wejściu w tryb zaznaczenia ukażą się dwie ikony: ikona usunięcia obrazu oraz ikona zmiany obrazu (Rysunek \ref{fig:trybZaznaczenia}).
Wybór pierwszej ikony skutkuje usunięciem obrazu wraz z akapitem, natomiast drugiej -- przejściem do pamięci urządzenia w celu wyboru zdjęcia.

\subsection{Listy}

Dostępne są różne rodzaje list. Różnią się one głównie etykietą oraz niewielkimi elementami, jak na przykład dostępne przekreślenie i zmiana koloru tekstu przy odznaczonej pozycji.

\subsubsection{Edycja}

Edycja list następuje poprzez interakcję użytkownika. Użytkownik może naciskać bezpośrednio na etykiety przy polu tekstowym w celu ich edycji (checkbox, licznik), jak również edytować tekst wiersza.

W przypadku niektórych typów wiersza możliwe jest jego odznaczenie. Wówczas tekst wiersza zmienia kolor na szary, a sam tekst zostaje przekreślony.

\subsubsection{Dodawanie wierszy}

Dodawanie wierszy następuje na dwa sposoby:

\begin{enumerate}
    \item W dowolnym miejscu listy poprzez dodanie znaku nowej linii.
    \item Na koniec listy poprzez kliknięcie przycisku "+" pod etykietami w trybie zaznaczenia. 
\end{enumerate}

Dodawanie wierszy poprzez znak nowej linii przenosi tekst na prawo od miejsca dodania znaku nowej linii do nowego wiersza.

\subsubsection{Usuwanie wierszy}

Usuwanie wierszy jest możliwe w trybie zaznaczania poprzez klikanie na etykiety ze znakiem "x".

\subsubsection{Zmiana typu wierszy}

Użytkownik może zmienić typ wszystkich wierszy w danej liście poprzez przejście w tryb zaznaczenia, a następnie w górym pasku nad listą (Rysunek \ref{fig:trybZaznaczenia}) wybrać jedną z dostępnych opcji.

\subsubsection{Dostępne typy wiersza}

\begin{itemize}
    \item checkbox;
    \item numerowane;
    \item oznaczone przez symbol "\--{}";
    \item naznaczane przez symbol "*" (w nowszej wersji aplikacji spodziewane jest tworzenie etykiet przez użytkownika);
    \item licznik.
\end{itemize}

Aktualnie listy są jednopoziomowe, bez możliwości zmiany wysunięcia od początku wiersza. Możliwość ta jest przewidziana w dalszym rozwoju aplikacji.

\subsubsection{Licznik}

Licznik służy do odliczania rzeczy opisywanej przez tekst wiersza. Z każdym kliknięciem zwiększana jest wartość licznika (lewa część etykiety), aż do osiągnięcia celu licznika (liczba w prawej części licznika). Wraz z osiągnięciem ustalonego celu wiersz zostaje odznaczony.

Możliwa jest zmiana wartości celu licznika. Aby to zrobić wystarczy nacisnąć na liczbę w prawej części licznika, a następnie wybrać liczbę na klawiaturze i zatwierdzić. Po zmianie następuje reset licznika.


\begin{figure}[ht]
    \centering
    \includegraphics[width=6cm]{images/liczniki.png}
    \caption{Edycja celu jednego z liczników.}
    \vspace{3mm}
\end{figure}