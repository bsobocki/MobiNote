

\chapter{Rozwój aplikacji MobiNote}
\label{ch:rozwoj}

\section{Struktura}

\subsection{Baza danych}

W kolejnych wersjach aplikacji MobiNote przewidziane jest rozszerzenie bazy danych o dodatkowe tabele:
\begin{compactitem}
    \item \textit{Notebooks} przechowującą wpisy dotyczące utworzonych zeszytów;
    \item \textit{NotesInNotebook} zawierającą wpisy dotyczące notatek w danych zeszytach.
\end{compactitem}

\subsection{Strona główna}

Na stronie głównej dostępne są przygotowane już przyciski \textbf{Trainings} oraz \textbf{Diet} reprezentujące przykładowy widok, który będzie używany w kolejnej wersji aplikacji MobiNote. Przyciski te reprezentują instancje typu \textbf{Notebook}, które będą odzwierciedleniem wpisów tabeli \textbf{Notebooks}.

Po naciśnięciu na przycisk użytkownik zostanie przeniesiony do strony z dostępnymi notatkami, będącymi częścią zeszytu reprezentowanego przez dany przycisk.

Przyciski w dolnym pasku strony głównej przygotowane są na przełączanie pomiędzy jej widokami. W obecnej aplikacji dostępny jest tylko widok główny, jednak przewidziany jest jeszcze widok zawierający tylko listę zeszytów (Library), oraz widok profilu użytkownika i jego ustawień (Profile).

\section{Rozszerzenie istniejących rozwiązań}

\subsection{Lista}

Klasa \textbf{NoteElementListWidget} posiada zmienną \verb|int depth| służącą do określania poziomu głębokości w liście. Ma to na celu określenie wielkości wcięcia, jak również używanych etykiet do oznaczania danego elementu (różne znaki, numeracja itd).

Dodatkowym elementem do edycji listy jest możliwość własnej definicji etykiet w liście. Będzie to ograniczone do pewnej głębokości. W zależności od głębokości, użytkownik będzie mógł zdefiniować symbol lub napis, który będzie używany jako etykieta.

\subsection{Widgety wewnętrzne w tekście}

Aktualnie w kodzie przygotowane są miejsca do parsowania widgetów wewnętrznych w tekście. Przygotowane są definicje wzorów w tekście, które miałyby zostać przekonwertowane w widgety.

Widgetami wewnętrznymi są między innymi:

\begin{compactitem}
    \item linki stron internetowych;
    \item linki notatek w bazie;
    \item inline latex -- do dodawania na przykład symboli matematycznych;
    \item zdjęcia (o rozmiarze nieprzekraczającym rozmiaru czcionki);
    \item cytowanie tekstu (zmiana koloru tła tekstu oraz czcionki).
\end{compactitem}

Użycie obiektów \textbf{InlineSpan} udostępnianych przez flutter pozwala tworzyć drzewiastą strukturę, której korzeniem jest \textbf{TextSpan}, natmiast dziećmi są inne obiekty \textbf{InlineSpan}, czyli instancje obiektów klas pochodnych \textbf{TextSpan} i \textbf{WidgetSpan}.
%https://api.flutter.dev/flutter/painting/InlineSpan-class.html

\subsection{Dodatkowe widgety w notatce}

Przygotowane również zostały obiekty typu \textbf{NoteWidgetData} oraz miejsca w \textbf{NoteListElementWidget} na dodatkowe elementy. Przykładem takiego elementu jest \textbf{NoteInfoPage}. Będzie on przyciskiem z ikoną, który po kliknięciu otwiera okno dialogowe z notatką/stroną informacyjną, którą będzie można przewijać.

Planowanymi dodatkowymi widgetami są między innymi:

\begin{compactitem}
    \item notatka głosowa;
    \item tabele.
\end{compactitem}

\section{Funkcjonalność}

\subsection{Alarmy}

Jedną z głównych funkcjonalności aplikacji MobiNote zaplanowaną na kolejne iteracje jest możliwość dodawania i ustawiania alarmów. Alarmy te będą używane w elementach list. Będą odmierzać czas i informować użytkownika o końcu przerwy, na przykład pomiędzy seriami ćwiczeń, bądź pomiędzy sesjami nauki.

\subsection{Powiadomienia}

Kolejną zaplanowaną funkcjonalnością jest dodanie powiadomień. Powiadomienia mają służyć jako przypomnienia ustawione w notatce dotyczące konkretnych działań. Przykładowo w notatce dotyczącej urodzin bliskiej osoby możemy ustawić przypomnienia o odebrania tortu, bądź prezentu. Nada to użyteczność notatce nawet, gdy nie będzie ona bezpośrednio w użyciu.

\subsection{Nagrywanie głosowej notatki}

Przydatną funkcjonalnością danej aplikacji będzie możliwość nagrywania notatek głosowych. Będą one przydatne w sytuacji, kiedy użytkownik będzie potrzebował na szybko zapisać daną treść i wrócić do niej w późniejszym czasie.

\subsection{Karty step-by-step}

Ostatnią funkcjonalnością przewidzianą w aplikacji MobiNote będą karty \textbf{step-by-step}.
Funkcjonalność ta będzie polegać na przygotowaniu listy rzeczy do zrobienia, aby osiągnąć zaplanowany cel. Następnie użytkownik zapisuje daną listę i przechodzi do widoku danej notatki. Widok ten składać się będzie tylko z jednego pola z opisem kroku, na którym aktualnie znajduje się użytkownik. Będą dostępne przyciski nawigacyjne pozwalające na:
\begin{compactitem}
    \item cofnięcie się o jeden krok;
    \item przejście do następnego kroku.
    
\end{compactitem}

Karty te mają na celu pokazać użytkownikowi jedynie następny krok, zamiast całej listy. Pozwala to na uniknięcie nadmiernego rozmyślania i poczucia przytłoczenia związanego z ilością rzeczy, jakie trzeba wykonać, aby osiągnąć zaplanowany cel.