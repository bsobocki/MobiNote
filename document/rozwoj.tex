

\chapter{Rozwój Aplikacji}

\section{Struktura Aplikacji}

\subsection{Baza danych}

\subsection{Strona główna}

\subsection{Edytor notatek}

\subsection{Dodatkowe strony}

\section{Rozszerzenie istniejących rozwiązań}

\subsubsection{Lista}

Klasa \textbf{NoteElementListWidget} posiada zmienną \verb|int depth| służącą do określania poziomu głębokości w liście. Ma to na celu określenie wielkości wcięcia, jak również uzywanych etykiet do oznaczania danego elementu(różne znaki, numeracja itd).

Dodatkowym elmenetem do edycji listy jest możliwość własnej definicji etykiet w liście. Będzie to ograniczone do pewnej głębokości. W zależności od głębokości, użytkownik będzie mógł zdefiniować symbol lub napis, który będzie używany jako etykieta.

\subsection{Widgety wewnętrzne w tekście}

Aktualnie w kodzie przygotowane są miejsca do parsowania widgetów wewnętrznych w tekście. Przygotowane są definicje wzorów w tekście, które miałyby zostać przekonwertowane w widgety.

Widgetami wewnętrzymi są między innymi:

\begin{compactitem}
    \item linki stron internetowych
    \item linki notatek w bazie
    \item inline latex -- do dodawania na przykład symboli matematycznych
    \item zdjęcia (o rozmiarze nieprzekraczającym rozmiaru czcionki)
    \item cytowanie tekstu(zmiana koloru tła tekstu oraz czcionki)
\end{compactitem}

Użycie obiektów \textbf{InlineSpan} udostępnianych przez flutter pozwala tworzyć drzewiastą strukturę, której korzeniem jest \textbf{TextSpan}, natmiast dziećmi są inne obiekty \textbf{InlineSpan}, czyli instancje obiektów klas pochodnych \textbf{TextSpan} i \textbf{WidgetSpan}.
%https://api.flutter.dev/flutter/painting/InlineSpan-class.html

\subsection{Dodatkowe Widgety w notatce}

Przygotowane również zostały obiekty typu \textbf{NoteWidgetData} oraz miejsca w \textbf{NoteListElementWidget} na dodatkowe elementy takie jak na przykład \textbf{NoteInfoPage}, która będzie widgetem złożonym z ikony, który po kliknięciu otwiera okno dialogowe ze przewijalną notatką/stroną informacyjną.

Planowanymi dodatkowymi widgetami są między innymi:

\begin{compactitem}
    \item notatka głosowa(zobacz \hyperlink{sec:glosowaNotatka}{Nagrywanie głosowej notatki}
    \item 
\end{compactitem}

\subsection{Interakcja widgetów}

\section{Funkcjonalność}

\subsection{Alarmy}

\subsection{Powiadomienia}

\subsection{Nagrywanie głosowej notatki}
\label{sec:glosowaNotatka}

\subsection{Karty Step-By-Step}

