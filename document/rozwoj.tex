

\chapter{Podsumowanie}

Stworzyłem aplikację MobiNote na moje własne potrzeby, według moich upodobań. Wygląd i funkcjonalność aplikacji zostały zaprojektowane z myślą o przystępności i wygodzie użytkowania. Jestem zadowolony z aktualnego rezultatu, dlatego używanie aplikacji w wersji, która została wydana wraz z napisaniem danej pracy jest dla mnie przyjemne. Podczas codziennego testowania i używania aplikacji napotykałem problemy, które rozwiązywałem wraz z dalszą implementacją, jak również odkrywałem, jakie funkcjonalności mogą być przydatne i wprowadzałem je. Przykładem takiej funkcjonalności jest możliwość zmiany rozmiaru zdjęcia, która nie została przewidziana podczas planowania aplikacji.

Praca nad rozwojem aplikacji pozwoliła mi na rozwinięcie umiejętności implementowania aplikacji mobilnych, której dotychczas nie posiadałem. Użycie wzorców projektowych oraz dobrych praktyk programowania pozwoliły mi odpowiednio zaprojektować strukturę aplikacji i kodu. Wykorzystałem wiedzę zdobytą podczas studiów między innymi na temat rozwoju oprogramowania, programowania obiektowego, tworzenia i operowania danymi przy użyciu bazie danych, inżynierii oprogramowania, jak również zdobytą umiejętność rozwiązywania problemów.

Aplikacji MobiNote używam od momentu ukończenia parsera, ponieważ już w samej wersji tekstowej aplikacja pozwalała na tworzenie przydatnych notatek i zapisków.
Zaprojektowany sposób interakcji z zawartością notatki sprawdził się podczas tworzenia notatek do treningu na siłowni, ponieważ pozwala na sprawne tworzenie i edycję list. Uważam, że dobrym pomysłem było zaimplementowanie własnego parsera, ponieważ podczas użytkowania aplikacji zauważyłem, że ujawnianie znaczników stylów i użytych nagłówków w akapitach tekstowych jest bardzo pomocne przy edycji tekstu i dostosowywaniu formatowania. Ustawianie typów wielkości nagłówka w akapitach okazało się również przydatne podczas tworzenia odstępów pomiędzy akapitami, ponieważ aplikacja umożliwia ustawienie wielkości nagłówka, który może się składać z samej spacji, dzięki czemu odstępy te mogą być zwiększane zgodnie z upodobaniami użytkownika.

Widget listy w różnych trybach sprawdził się podczas różnych aktywności: od treningu na siłowni (lista typu licznik) po robienie zakupów (lista typu checkbox), natomiast dodawanie obrazów pozwala na rozszerzenie notatek o przydatne schematy, czy poprawę wyglądu. Jest to kolejny powód, dzięki któremu pomimo swojej prostoty aplikacja ma dla mnie charakter użytkowy, a nie tylko teoretyczny na potrzeby pracy.

W wykonaniu aplikacji dostrzegam również drobne błędy i niedociągnięcia, jak na przykład niekontrolowane ukrywanie i ujawnianie klawiatury podczas przewijania ekranu, lub dodawania nowego akapitu tekstowego. Dzieje się tak, ponieważ przy tworzeniu nowych obiektów typu \textbf{TextField} zgłaszana jest jego aktywność (focus) oraz przebudowywany jest stan (state) widgetu głównego widoku \textbf{ContentEditor}. Przy dalszym rozwoju aplikacji niezbędne byłoby użycie menadżera stanu \cite{sateManagement}, aby lepiej zorganizować odświeżanie widoku i aktualizowanie stanu widgetów.

Podczas pracy nad aplikacją narodziły się pomysły na rozwój istniejących rozwiązań, natomiast podczas użytkowania aplikacji w różnych sytuacjach rósł apetyt na kolejne funkcjonalności. Pomysły te omówię w następnym rozdziale.

\chapter{Rozwój aplikacji MobiNote}
\label{ch:rozwoj}

\section{Struktura}

\subsection{Baza danych}

W kolejnych wersjach aplikacji MobiNote przewidziane jest rozszerzenie bazy danych o dodatkowe tabele:
\begin{compactitem}
    \item \textit{Notebooks} przechowującą wpisy dotyczące utworzonych zeszytów;
    \item \textit{NotesInNotebook} zawierającą wpisy dotyczące notatek w danych zeszytach.
\end{compactitem}

\subsection{Strona główna}

Na stronie głównej dostępne są przygotowane już przyciski \textbf{Trainings} oraz \textbf{Diet} reprezentujące przykładowy widok, który będzie używany w kolejnej wersji aplikacji MobiNote. Przyciski te reprezentują instancje typu \textbf{Notebook}, które będą odzwierciedleniem wpisów tabeli \textbf{Notebooks}.

Po naciśnięciu na przycisk użytkownik zostanie przeniesiony do strony z dostępnymi notatkami, będącymi częścią zeszytu reprezentowanego przez dany przycisk.

Przyciski w dolnym pasku strony głównej przygotowane są na przełączanie pomiędzy jej widokami. W obecnej aplikacji dostępny jest tylko widok główny, jednak przewidziany jest jeszcze widok zawierający tylko listę zeszytów (Library), oraz widok profilu użytkownika i jego ustawień (Profile).

\section{Rozszerzenie istniejących rozwiązań}

\subsection{Lista}

Klasa \textbf{NoteElementListWidget} posiada zmienną \verb|int depth| służącą do określania poziomu głębokości w liście. Ma to na celu określenie wielkości wcięcia, jak również używanych etykiet do oznaczania danego elementu (różne znaki, numeracja itd).

Dodatkowym elementem do edycji listy jest możliwość własnej definicji etykiet w liście. Będzie to ograniczone do pewnej głębokości. W zależności od głębokości, użytkownik będzie mógł zdefiniować symbol lub napis, który będzie używany jako etykieta.

\subsection{Widgety wewnętrzne w tekście}

Aktualnie w kodzie przygotowane są miejsca do parsowania widgetów wewnętrznych w tekście. Przygotowane są definicje wzorów w tekście, które miałyby zostać przekonwertowane w widgety.

Widgetami wewnętrznymi są między innymi:

\begin{compactitem}
    \item linki stron internetowych;
    \item linki notatek w bazie;
    \item inline latex -- do dodawania na przykład symboli matematycznych;
    \item zdjęcia (o rozmiarze nieprzekraczającym rozmiaru czcionki);
    \item cytowanie tekstu (zmiana koloru tła tekstu oraz czcionki).
\end{compactitem}

Użycie obiektów \textbf{InlineSpan} udostępnianych przez flutter pozwala tworzyć drzewiastą strukturę, której korzeniem jest \textbf{TextSpan}, natmiast dziećmi są inne obiekty \textbf{InlineSpan}, czyli instancje obiektów klas pochodnych \textbf{TextSpan} i \textbf{WidgetSpan}.
%https://api.flutter.dev/flutter/painting/InlineSpan-class.html

\subsection{Dodatkowe widgety w notatce}

Przygotowane również zostały obiekty typu \textbf{NoteWidgetData} oraz miejsca w \textbf{NoteListElementWidget} na dodatkowe elementy. Przykładem takiego elementu jest \textbf{NoteInfoPage}. Będzie on przyciskiem z ikoną, który po kliknięciu otwiera okno dialogowe z notatką/stroną informacyjną, którą będzie można przewijać.

Planowanymi dodatkowymi widgetami są między innymi:

\begin{compactitem}
    \item notatka głosowa;
    \item tabele.
\end{compactitem}

\section{Funkcjonalność}

\subsection{Alarmy}

Jedną z głównych funkcjonalności aplikacji MobiNote zaplanowaną na kolejne iteracje jest możliwość dodawania i ustawiania alarmów. Alarmy te będą używane w elementach list. Będą odmierzać czas i informować użytkownika o końcu przerwy, na przykład pomiędzy seriami ćwiczeń, bądź pomiędzy sesjami nauki.

\subsection{Powiadomienia}

Kolejną zaplanowaną funkcjonalnością jest dodanie powiadomień. Powiadomienia mają służyć jako przypomnienia ustawione w notatce dotyczące konkretnych działań. Przykładowo w notatce dotyczącej urodzin bliskiej osoby możemy ustawić przypomnienia o odebrania tortu, bądź prezentu. Nada to użyteczność notatce nawet, gdy nie będzie ona bezpośrednio w użyciu.

\subsection{Nagrywanie głosowej notatki}

Przydatną funkcjonalnością danej aplikacji będzie możliwość nagrywania notatek głosowych. Będą one przydatne w sytuacji, kiedy użytkownik będzie potrzebował na szybko zapisać daną treść i wrócić do niej w późniejszym czasie.

\subsection{Karty step-by-step}

Ostatnią funkcjonalnością przewidzianą w aplikacji MobiNote będą karty \textbf{step-by-step}.
Funkcjonalność ta będzie polegać na przygotowaniu listy rzeczy do zrobienia, aby osiągnąć zaplanowany cel. Następnie użytkownik zapisuje daną listę i przechodzi do widoku danej notatki. Widok ten składać się będzie tylko z jednego pola z opisem kroku, na którym aktualnie znajduje się użytkownik. Będą dostępne przyciski nawigacyjne pozwalające na:
\begin{compactitem}
    \item cofnięcie się o jeden krok;
    \item przejście do następnego kroku.
    
\end{compactitem}

Karty te mają na celu pokazać użytkownikowi jedynie następny krok, zamiast całej listy. Pozwala to na uniknięcie nadmiernego rozmyślania i poczucia przytłoczenia związanego z ilością rzeczy, jakie trzeba wykonać, aby osiągnąć zaplanowany cel.