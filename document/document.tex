% Opcje klasy 'iithesis' opisane sa w komentarzach w pliku klasy. Za ich pomoca
% ustawia sie przede wszystkim jezyk oraz rodzaj (lic/inz/mgr) pracy.
\documentclass[shortabstract]{iithesis}

\usepackage[utf8]{inputenc}

%%%%% DANE DO STRONY TYTUŁOWEJ
% Niezaleznie od jezyka pracy wybranego w opcjach klasy, tytul i streszczenie
% pracy nalezy podac zarowno w jezyku polskim, jak i angielskim.
% Pamietaj o madrym (zgodnym z logicznym rozbiorem zdania oraz estetyka) recznym
% zlamaniu wierszy w temacie pracy, zwlaszcza tego w jezyku pracy. Uzyj do tego
% polecenia \fmlinebreak.
\polishtitle    {Aplikacja mobilna do zarządzania i tworzenia\fmlinebreak interaktywnych notatek}
\englishtitle   {A mobile application to manage and create interactive notes}
\polishabstract {TODO: streszczenie po polsku\ldots}
\englishabstract{TODO: streszczenie po angielsku\ldots}
% w pracach wielu autorow nazwiska mozna oddzielic poleceniem \and
\author         {Bartosz Sobocki}
% w przypadku kilku promotorow, lub koniecznosci podania ich afiliacji, linie
% w ponizszym poleceniu mozna zlamac poleceniem \fmlinebreak
\advisor        {dr Marcin Młotkowski}
%\date          {}                     % Data zlozenia pracy
% Dane do oswiadczenia o autorskim wykonaniu
%\transcriptnum {}                     % Numer indeksu
%\advisorgen    {dr. Marcina Młotkowskiego} % Nazwisko promotora w dopelniaczu
%%%%%

%%%%% WLASNE DODATKOWE PAKIETY
%
%\usepackage{graphicx,listings,amsmath,amssymb,amsthm,amsfonts,tikz}
%
%%%%% WŁASNE DEFINICJE I POLECENIA
%
%\theoremstyle{definition} \newtheorem{definition}{Definition}[chapter]
%\theoremstyle{remark} \newtheorem{remark}[definition]{Observation}
%\theoremstyle{plain} \newtheorem{theorem}[definition]{Theorem}
%\theoremstyle{plain} \newtheorem{lemma}[definition]{Lemma}
%\renewcommand \qedsymbol {\ensuremath{\square}}
% ...
%%%%%

\begin{document}

%%%%% POCZĄTEK ZASADNICZEGO TEKSTU PRACY

\chapter{Wprowadzenie}

\section{Cel projektu}

Celem projektu jest wytworzenie aplikacji mobilnej do zarządzania notatkami w interaktywny sposób.
Aplikacja ma umożliwić użytkownikowi prowadzenie i tworzenie notatek pozwalających na edycję tekstu i inncyh elementów znajdujących się w notatce bez konieczności zmian widoków, przełączenia całej zawartości pomiędzy trybem edycji, a trybem użytkowym, czy potrzeby zapisu notatki, aby jej zawartość była dostępna w postaci końcowego efektu.
Graficzny interfejs użytkownika powinien być przystępny i pozwalać na edycję poszczególnych elementów, podczas gdy pozostała zawartość notatki jest wyświetlana w trybie widoku.
Do przechowywania zawartości notatek powinna zostać użyta baza danych, jak również odpowiedni format zapisu dancyh do przechowywania zawartości niebędącej tekstem. 

\section{Opis Aplikacji}

Aplikacja MobiNote służy do zarządzania notatkami w sposób opisany powyżej.
Głównym zamysłem aplikacji jest pomoc użytkownikowi w prowadzeniu, tworzeniu i zapisywaniu notatek, w których skład wchodzi nie tylko tekst, ale również przyciski, obrazy, listy, liczniki, alarmy, powiadomienia i różnego rodzaju pomocne widgety.
Aplikacja umożliwia tworzenie i utrzymywanie brudnopisu używanego podczas codziennych czynności, jak również przejrzystych, dopracowanych i przystępnych  notatek.

Wytworzone notatki mogą być używane w formie brudnopisu, między innymi podczas: treningu na siłowni, organizacji przyjęcia, nauki z wykorzystaniem sesji pomodoro, tworzeniu listy zakupów oraz wielu innych codziennych czynności.

Mogą także w przyjemny dla oka sposób przechowywać i wyświetlać informacje przygotowane w celu tworzenia dziennika, notatek do nauki, spisu pomysłów i ważnych myśli, czy tworzenia różnego rodzaju list i opisów, takich jak lista miejsc, które użytkownik chciałby odwiedzić wraz z opisem i zdjęciami miejsc, które chciałby tam zobaczyć.

Interaktywność notatki jest zapewniana poprzez możliwość tworzenia zawartości na bierząco za pomocą klawiatury i dostępnego interfejsu użytkownika.
Uzytkownik może tworzyć styl tekstu za pomocą znaczników dodawanych wewnątrz tekstu w odpowiednich miejscach, podobnie do języka znaczników Markdown.
Dodawając odpowiednie znaczniki w tekście można ustawić wielkość czcionki w danym paragrafie, kursywe, podkreślenie, przekreślenie, a także pogrubienie.
Tekst jest automatycznie dostosowywany wraz z dodaniem znaczników, co pozawala na bierząco obserwować i dostosowywać style i wielkości czcionki do potrzeb użytkownika.

Interfejs użytkownika jest prosty i przejrzysty, posiada motyw ciemny(dark), jasny(light) oraz ułatwiający użytkowanie(easy) dla osób widzących słabiej.
\chapter{Implementacja}

%%%%% BIBLIOGRAFIA

%\begin{thebibliography}{1}
%\bibitem{example} \ldots
%\end{thebibliography}

\end{document}