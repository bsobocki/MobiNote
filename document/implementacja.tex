\chapter{Implementacja}
    
Implementując aplikację MobiNote starałem się zachowywać czysty kod, podzielony na jak najmniejsze części pod względem odpowiedzialności i zadań klas oraz metod, jak również czyste repozytorium zachowując odpowiedni podział i strukturę katalogów, modułów i pozostałych plików. Przykładałem dużą wagę do jakości oraz wykonania, aby powrót do poszczególnych warstw i miejsc w aplikacji był prosty, a sama jej struktura była przejrzysta i zrozumiała. 

\section{Technologie}

\subsection{flutter}

Aplikacja została napisana przy użyciu frameworka flutter w języku Dart.

Wybierając technologie kierowałem się kryteriami takimi jak: prostota, wieloplatformowość, estetyka bez dużego wkładu w kreowanie komponentów na własną rękę oraz wydajność. Jedną z najbardziej polecanych framework'ów na stronie  linkedin.com \cite{linkedin} oraz itCraft \cite{itCraft} był flutter. Jak podaje strona główna flutter.dev \cite{flutter}:

"Flutter is an open source framework by Google for building beautiful, natively compiled, multi-platform applications from a single codebase."

Składnia języka Dart jest stosunkowo prosta i przejrzysta, natomiast framework flutter pozwala na programowanie aplikacji mobilnych w prosty sposób na platformy IOS oraz Android o dużej wydajności zważywszy na to, że język Dart jest kompilowany do natywnego kodu. Pozwala to na tworzenie aplikacji szybko i bez wielkiego nakładu pracy, dając przy tym duże możliwości i przyjemny dla oka rozbudowany interfejs użytkownika.

\subsection{SQLite}

Do przechowywania notatek została wykorzystana baza danych SQLite. Aplikacja wykorzystuje bibliotekę sqlite3 \cite{sqlite3}, umożliwiającą wykonywanie operacji na lokalnej bazie danych SQLite. Pozwala ona na zapis, odczyt a także manipulację danymi. Dodatkowo wykorzystywany jest pakiet drift \cite{drift}, który jest rodzajem systemu ORM, umożliwiający mapowanie między obiektami języka Dart oraz tabelami bazy danych SQLite. Pozwala to na pracę bezpośrednio w kodzie Dart używając dostępnej funkcjonalności bez konieczności pisania zapytań bezpośrednio w języku SQL.

\section{Organizacja repozytorium}

Kod aplikacji został podzielony na moduły i zorganizowany w zależności od poziomu abstrakcji i zastosowań.

Główny katalog z kodem źródłowym \textbf{lib} zawiera:

\begin{itemize}
    \item plik \textbf{main.dart} z wywołaniem głównej funkcji main budującej aplikację;
    \item katalog \textbf{database};
    \item katalog \textbf{logic};
    \item katalog \textbf{screens}.
\end{itemize}

\subsubsection{database}

Zawiera definicję bazy danych, struktury oraz metody z nią związane. Posiada wygenerowany na podstawie pliku \textbf{database\_{}def.dart} plik \textbf{database\_{}def.g.dart} zawierający struktury odzwierciedlające tabele bazy danych w klasy języka Dart. 

\subsubsection{logic}

Znajdują się tam definicje struktur reprezentujących stan komponentów, definicje typów (stylów tekstu, widgetów, znaczników itd.), mapowanie kluczy (String) na style tekstu, znaczniki i elementy, logika parsera, funkcje pomocnicze wraz z funkcjonalnością aplikacji niebędącej bezpośrednią częścią widgetów.

\subsubsection{screens}

Dostępne są tam definicje stron wraz z funkcjonalnością i metodami ich komponentów, jak również definicje motywów i kolorów aplikacji.

\section{Wykorzystane rozwiązania}

\subsubsection{Programowanie Obiektowe}

Dart promuje programowanie obiektowe, dlatego też w projekcie został zastosowany paradygmat obiektowego programowania wraz z jego mechanizmami. Przykładami tych mechanizmów może być wielokrotne zastosowanie abstrakcji różnych poziomów, dziedziczenie i polimorfizm.

Przykładem zastosowania wszystkich powyższych mechanizmów jest reprezentacja, tworzenie oraz zarządzanie widgetami. 
Dzięki zastosowaniu opisanych mechanizmów możliwe było zaimplementowanie fabryki, listy, oraz akapitów trzymanych w edytorze.

\subsubsection{Wzorce projektowe}

W powyższych opisach możemy zauważyć użycie wzorców projektowych spotykanych podczas implementacji aplikacji mobilnej \cite{flutterdesignpatterns}. Są to między innymi:

\begin{itemize}
    \item \textbf{Fabryka} -- użyty w przypadku fabryki widgetów \textit{NoteEditorWidgetFactory};
    \item \textbf{Singleton} -- używany w przypadku dostępu do bazy danych oraz dostępu do ustawionego motywu aplikacji (\textit{MobiNoteDatabase}, \textit{MobiNoteTheme});
    \item \textbf{Kompozyt} -- używany w przypadku struktur drzewiastych takich jak na przykład obiekty \textit{SpanTree}, czy \textit{NoteWidgetData};
\end{itemize}

\subsubsection{Testy Jednostkowe}

Część logiki niezwiązanej bezpośrednio z wyświetlaniem i organizacją komponentów i ich wyglądu testowałem za pomocą testów jednostkowych od razu podczas ich implementacji przy użyciu biblioteki testing \cite{testing}. Dzięki temu uniknąłem długich godzin spędzonych na poszukiwaniu problemów, gdy któryś z komponentów mógł nie działać poprawnie. Przyczyną niepoprawnego działania może być nie tylko algorytm, ale i sam widget i funkcjonowanie frameworka flutter. Dostosowując szczegółowość testów jednostkowych starałem się pokryć przypadki użycia jeszcze przed użyciem danych klas, metod i algorytmów w aplikacji. Przetestowany został w ten sposób parser języka znaczników stylów, wraz z pełną funkcjonalnością, jak również konwersja reprezentacji widgetów do formatu JSON.

\subsection{Baza danych}

Do zarządzania dostępem i manipulacją bazy danych tworzony jest jeden obiekt typu \textbf{MobiNoteDatabase} dla całego projektu. Jest on tworzony i udostępniany za pomocą biblioteki do zarządzania stanem \textbf{GetX} \cite{getx}. 

W implementacji aplikacji MobiNote, \textbf{GetX} pozwala na tworzenie instancji zarządzającej dostępem do bazy danych raz i udostępnianie jej w różnych miejscach aplikacji. Struktura aplikacji wymaga tylko jednej instancji obiektu \textbf{MobiNoteDatabase}, dlatego tworzone jest jedno połączenie w jej konstruktorze, a sam obiekt przechowywany jest za pomocą \textbf{GetX}.

\begin{verbatim}
    Get.put(MobiNoteDatabase());
\end{verbatim}

\noindent Następnie za pomocą funkcji \textbf{find} może zostać udostępniany

\begin{verbatim}
    final database = Get.find<MobiNoteDatabase>();
\end{verbatim}

Umożliwia to zwiększenie wydajności, ponieważ unikamy otwierania połączenia z bazą danych za każdym razem, gdy jest ono potrzebne (obie strony: strona główna, oraz edytor notatki używają połączenia z bazą danych do swoich funkcjonalności). Używany zatem jest tutaj wzorzec Singleton. Z racji tego, że nie ma potrzeby otwierania równoległych połączeń (używana jest tylko jedna ze stron i ich funkcjonalność naraz) wzorzec ten został zastosowany w implementacji.

\subsection{Dobre praktyki}

\subsubsection{Leniwe tworzenie widgetów w ListView}

Podczas implementacji listy notatek w stronie głównej oraz edytora zawartości notatki użyty został widget \textbf{ListView} poprzez wywołanie tzw. \textit{named constructor} \textbf{ListView.builder}\cite{listviewbuilder}. Praktyka ta ma dobre zastosowanie przy długich listach, ponieważ używając konstruktora \textbf{builder} widgety zawarte w \textbf{ListView} budowane są w sposób leniwy, na bieżąco, jeśli istnieje potrzeba ich wyświetlenia. Z racji tego, że budowanie jest szybkie i nieskomplikowane pozwala to uniknąć przechowywania dużej ilości widgetów (znajduje to pozytywne skutki przy długich, skomplikowanych notatkach).
%https://docs.flutter.dev/cookbook/lists/long-lists

\subsubsection{Reprezentacja widgetów}

Każdy widget reprezentowany jest poprzez obiekt typu pochodnej klasy \textbf{WidgetData} przechowywującej parametry danego widgetu potrzebne do jego prawidłowego wyświetlania oraz funkcjonowania. Oddzielana jest warstwa wyświetlania od warstwy informacji danego widgetu. Pozwala to na zapis stanu widgetu do bazy danych, ponieważ zawarta w obiektach reprezentujących warstwę informacji jest konwersja do formatu JSON, który jest przechowywany w bazie danych jako fragment zawartości danej notatki.

\subsubsection{ID Generator}

Widgety zawierają identyfikatory do odróżniania ich w rodzicu lub liście zwierającej je. Ma to zastosowanie m.in. w znajdowaniu miejsca w liście np. do dodawania nowego elementu, czy usuwania i zmiany elementów. Lepiej nadać \texttt{id} i na podstawie znalezienia \texttt{id} na liście pobierać indeks, zamiast nadawać i aktualizować indeksy elementom przy każdej edycji listy.

Do generowania identyfikatorów w odpowiednich miejscach w kodzie służy obiekt klasy \textbf{IdGenerator}.

\subsubsection{Fabryka widgetów}

Widgety tworzone są na podstawie obiektów reprezentacyjnych (opisanych powyżej) w klasie \textbf{NoteEditorWidgetFactory}. Na podstawie pola \texttt{type} wybierany jest typ widgetu, który następnie tworzony jest z parametrów obiektu reprezentującego. Używa klasy \textbf{IdGenerator} do generowania identyfikatorów, przez co obsługa przydzielania \texttt{id} jest robiona w jednym miejscu, w fabryce, która jest przekazywana dzieciom i elementom danych widgetów. Pozwala to zachować spójność w aktualnej implementacji, jak również przy dalszym rozwoju.